\documentclass[a4paper,debug]{tufte-handout}

\def\TheTitle{Code Loops: There and Back Again}

\title{\TheTitle}
\date{\today}
\author{Ben Nagy and David Michael Roberts}

\usepackage{amsmath,amsthm}
\usepackage{url}

\hypersetup{
urlcolor = red,
colorlinks = true,
linkcolor = blue,
citecolor = blue,
linktocpage = true,
pdftitle = {\TheTitle},
pdfauthor = {Ben Nagy, David M. Roberts},
bookmarksopen = false,
bookmarksopenlevel = 1,
unicode = true,
hypertexnames =false
}%

\DeclareMathOperator{\Hamming}{Hamming}
\DeclareMathOperator{\dom}{dom}
\DeclareMathOperator{\Span}{span}
\newcommand{\F}{\mathbb{F}}

\theoremstyle{plain}
\newtheorem{theorem}{Theorem}
\newtheorem{lemma}[theorem]{Lemma}
\newtheorem{proposition}[theorem]{Proposition}
\newtheorem{corollary}[theorem]{Corollary}

\theoremstyle{definition}
\newtheorem{definition}[theorem]{Definition}

\theoremstyle{remark}
\newtheorem{example}[theorem]{Example}
\newtheorem{remark}[theorem]{Remark}

\setlength{\parindent}{0pt}
\begin{document}

\maketitle

\medskip

\subsection{Abstract}
\noindent This paper will describe an implementation of an algorithm by Griess to construct a twisted cocycle describing the multiplication in a code loop, as well as a method to reconstruct the full cocycle from a function defined on a domain of size approximately the square root of the original. 
\marginnote{Every sentence should start on a new line in the source}Additionally, we find a new basis for the extended Golay code with the property that the twisted cocycle built from it by Griess' algorithm is trivial on a pair of subspaces with trivial intersection, of dimensions 6 and 5 respectively.


\section{Intro}
Recall the concept of \emph{code loop} from Griess\cite{Griess}, and also that of a \emph{factor set}, which we will call a \emph{twisted cocycle}. 
Let $C \subset \F_2^{4n}$ be a doubly-even code.\sidenote{Here is a sidenote}
The twisted cocycle identity for $\theta\colon C\times C \to \F_2$ is
\[
	\theta(x,y) + \theta(x,y+z) + \theta(x+y,z) + \theta(y,z) = |x\wedge y \wedge z|
\]
where $\wedge$ is co\"ordinate-wise multiplication, and $|\cdot|$ is the bit weight of a vector, namely the number of non-zero entries.\marginnote{Here is a marginnote}

\section{Griess' algorithm}

Here's a sample of the first pass of the algorithm\sidenote{I guess we should use some kind of algorithm package to write this up? Eg \url{https://en.wikibooks.org/wiki/LaTeX/Algorithms#Typesetting_using_the_algorithmicx_package}}

\begin{itemize}

	\item[D0:] $\theta(v_0,v_0) = |v_0|/4,\quad \theta(0,0) = \theta(0,v_0) = \theta(v_0,0) = 0$.

	\item[D1:] For all $x \in \Span\{v_0\}$, 
	\begin{itemize}
		\item[] $\theta(v_1,x) = 0$, \marginnote{arbitrary choice, but with $\theta(v_1,0)=0$}
		\item[] $\theta(x,v_1) = |x\cap v_1|/2$.
	\end{itemize}

	\item[D2:] For all $x \in \Span\{v_0\}$,
	\begin{itemize}
	 	\item[] $\theta(v_1,v_1+x) = |v_1|/4$,
	 	\item[] $\theta(v_1 + x,v_1) = |v_1|/4 + |v_1\cap(v_1+x)|/2$.
	 \end{itemize} 
	 \item[D3:] For all $x,y\in \Span\{v_0\}$,
	 \begin{itemize}
	 	\item[] $\theta(v_1+x,v_1+y) = |y\cap(v_1+x)|/2 + |y\cap v_1|/2 + \theta(x,y) + |v_1|/4 + |v_1\cap(v_1+x)|/2$.
	 \end{itemize}
	 \item[D4:] For all $x,y\in \Span\{v_0\}$,
	 \begin{itemize}
	 	\item[] $\theta(v_1+x,y) = |v_1+x|/4 + \theta(v_1+x,v_1+x+y)$,
	 	\item[] $\theta(y,v_1+x) = |y\cap(v_1+x)|/2 + \theta(v_1+x,y)$.
	 \end{itemize}

\end{itemize} 

\section{Implementation}

All kinds of crazy bit-level tricks...

Benchmarking etc.

\section{Reducing the size of the domain} 
We can split $C$ as $V\oplus W$ and denote the restriction of $\theta$ to $(V\cup W)\times (V \cup W)$ by $\alpha$, which is a function
\[
	V\oplus V \cup W\oplus W \cup V\oplus W \cup W\oplus V \to \F_2.
\]

\begin{proposition}\label{eq:reconstructing theta}

We can write $\theta$ entirely in terms of $\alpha$ as follows:
\begin{align*}
	\theta(v_1+w_1,v_2+w_2)	& = \alpha(v_1,v_2)  + \alpha(w_1,w_2) + \alpha(v_1,w_1) + \alpha(w_2,v_2) + \alpha(v_1+v_2,w_1+w_2)\\
							& + \frac12|v_2\wedge(w_1+w_2)| + |v_1\wedge v_2 \wedge (w_1+w_2)| + |w_1\wedge w_2 \wedge v_2| + |v_1\wedge w_1 \wedge (v_2 + w_2)| 
\end{align*}
\end{proposition}

Note that the domain of $\alpha$ has size $(2^k + 2^l - 1)^2$, whereas the domain of $\theta$ has size $2^{2(k+l)}$, where $k=\dim V$ and $l=\dim W$.
If $k=l$, then $|\dom(\alpha)| = (2^{k+1}-1)^2 \approx \sqrt{|\dom(\theta)|} = 2^{4k}$.

So if $\theta$ has been already calculated using the algorithm in the section `Griess' algorithm', then one only needs to store $\alpha$, and use Proposition~\ref{eq:reconstructing theta} to calculate any other values on the fly.

\section{Pretty pictures}

Go here

\bibliographystyle{alpha}
\bibliography{bib}
\end{document}